\documentclass[12pt, a4paper]{article}
\usepackage[utf8]{inputenc}
\usepackage{lmodern}
\usepackage[top=1in, bottom=1in, left=1.25in, right=1.25in]{geometry}
\usepackage{titlesec}
\usepackage{hyperref}
\usepackage{graphicx}
\usepackage{setspace}
\usepackage{amsmath} % Required for math and equations
\usepackage[style=ieee]{biblatex} % Mandatory IEEE Citation Style
\usepackage{tocloft}
\usepackage[printonlyused]{acronym} % For auto-updating List of Abbreviations

% --- 1. Table of Contents & Lists Global Styling ---
% --- 1. Table of Contents & Lists Global Styling ---
\makeatletter
\renewcommand{\@dotsep}{2} 
\makeatother

\renewcommand{\cftsecdotsep}{\cftdotsep} 
\renewcommand{\cftsecleader}{\cftdotfill{\cftdotsep}} 
\renewcommand{\cftsecfont}{\bfseries} 
\renewcommand{\cftsecpagefont}{\bfseries}

% --- SPACING SETTINGS ---
% 1. Space between Section number "1.0" and title "Introduction"
\setlength{\cftsecnumwidth}{3.0em} 

% 2. Vertical space between Section (1.0) and Subsection (1.1)
\setlength{\cftbeforesecskip}{1.0em}    % Space above "1.0 Introduction"
\setlength{\cftbeforesubsecskip}{0.4em} % Space above "1.1 Problem Statement"

% 3. Indentation for Subsections
\setlength{\cftsecindent}{0pt}
\setlength{\cftsubsecindent}{3.0em} 
\renewcommand{\cftsubsecfont}{\normalfont}

% --- 2. Custom Font Sizes for Main Body ---
% Section (1.0, 2.0) - 16pt, Bold, Left Aligned
\titleformat{\section}
  {\normalfont\fontsize{16pt}{20pt}\bfseries}{\thesection}{1em}{}

% Subsection (1.1, 2.1) - 14pt, Bold, Left Aligned
\titleformat{\subsection}
  {\normalfont\fontsize{14pt}{18pt}\bfseries}{\thesubsection}{1em}{}

% --- 3. Section Numbering Format (X.0) ---
\renewcommand{\thesection}{\arabic{section}.0}
\renewcommand{\thesubsection}{\arabic{section}.\arabic{subsection}}

% --- 4. List of Equations Auto-Update Setup ---
\newcommand{\listequationsname}{List of Equations}
\newlistof{equations}{equ}{\listequationsname}
\newcommand{\myequation}[1]{%
    \addcontentsline{equ}{equations}{\protect\numberline{\theequation}#1}%
}

\addbibresource{references.bib}

\begin{document}

% --- PART 1: FRONT MATTER (Roman Numerals) ---
\pagenumbering{roman} 
\setcounter{page}{1}

% Page i: Title Page
\input{titlepage} 
\clearpage

% Pages ii - vii: Standard Front Matter
\input{frontmatter/preface.tex}         % Page ii
\input{frontmatter/tableofcontents.tex}  % Page iii
\input{frontmatter/listoffigures.tex}    % Page iv
\input{frontmatter/listoftables.tex}     % Page v
% --- List of Equations Page ---
\clearpage
\thispagestyle{plain} % Shows the Roman numeral 'vi' at the bottom

% Centers your custom title
\begin{center}
    \textbf{\Large List of Equations}
\end{center}
\vspace{0.5cm}

% Use a group to locally hide the default \listofequations title
\begingroup
\renewcommand{\listequationsname}{} % This removes the second "List of Equations" title
\listofequations 
\endgroup

% Adds the "LIST OF EQUATIONS" entry to the Table of Contents
\addcontentsline{toc}{section}{LIST OF EQUATIONS}
\newpage  % Page vi
% --- List of Abbreviations Page ---
\clearpage
\thispagestyle{plain} % Shows the Roman numeral 'vii' at the bottom
\addcontentsline{toc}{section}{LIST OF ABBREVIATIONS}

\begin{center}
    \fontsize{16pt}{20pt}\selectfont\textbf{List of Abbreviations}
\end{center}
\vspace{0.5cm}

\begin{acronym}[RNNNNN] % The bracket [RNNNNN] sets the indentation width
    \acro{RNN}{Recurrent Neural Network}
    \acro{LSTM}{Long Short-Term Memory}
    \acro{EER}{Equal Error Rate}
    \acro{FAR}{False Acceptance Rate}
    \acro{FRR}{False Rejection Rate}
    \acro{IEEE}{Institute of Electrical and Electronics Engineers}
\end{acronym}    % Page vii

% --- PART 2: MAIN CONTENT (Arabic Numerals) ---
\clearpage
\pagenumbering{arabic}
\setcounter{page}{1} % Introduction starts at Page 1

% University of Peradeniya usually requires 1.5 or double spacing
\onehalfspacing 

\section[INTRODUCTION]{Introduction}

\paragraph{Overview of the Research Domain}
In mechanisms such as \ac{PIN} and passwords are increasingly becoming single points of failure \cite{pirzado2021keystroke}. They are susceptible to vulnerabilities like shoulder surfing, brute-force attacks, and social engineering \cite{pirzado2021keystroke, kiyani2020continuous}. To address these vulnerabilities, Behavioral Biometrics---specifically Keystroke Dynamics and Mouse/Touch Dynamics---has emerged as a powerful alternative \cite{rahman2021scalable, kim2024kdprint}. Unlike static passwords, behavioral biometrics allow for continuous, passive authentication, verifying a user's identity based on \textit{how} they interact with a device rather than \textit{what} they know \cite{shepherd1995continuous, monrose1999hardening}.

\subsection{Problem Statement}
The primary problem addressed by this research is the critical trade-off between \textbf{user privacy} and \textbf{system latency} in behavioral authentication systems on \textbf{resource-constrained edge devices}. While behavioral biometrics---such as keystroke dynamics and touch interactions---offer a robust solution for continuous authentication, the storage and processing of these behavioral patterns present significant security risks \cite{rahman2021scalable}.

This issue affects millions of users across \textbf{heterogeneous platforms, ranging from high-end desktops to resource-constrained mobile and IoT devices}. Unlike passwords, behavioral biometrics are immutable; a user cannot change their typing rhythm or hand geometry if the biometric template is compromised. Therefore, a breach of raw behavioral data constitutes a permanent loss of digital identity \cite{pirzado2021keystroke}.

Current approaches fail to address this problem effectively due to a technical dichotomy between accuracy and efficiency:

\begin{itemize}
    \item \textbf{Privacy Gaps in High-Accuracy Models:} State-of-the-art \ac{DL} frameworks, such as those utilizing \acp{RNN} or Image-based Encoding (e.g., KDPrint), achieve low \ac{EER} but typically require the storage of \textbf{raw behavioral features} \cite{kim2024kdprint, kiyani2020continuous}. This creates a single point of failure where the database becomes a high-value target for attackers.
    
    \item \textbf{Efficiency Gaps in Privacy-Preserving Methods:} Conversely, strong cryptographic solutions like \ac{FHE} allow for secure computation but suffer from prohibitive computational overhead. Research indicates that such heavy \ac{HE} schemes often introduce latencies ranging from seconds to minutes, rendering them impractical for \textbf{real-time, continuous authentication} where decisions must be made in milliseconds to maintain a seamless user experience \cite{rahman2021scalable}. \textbf{Therefore, applying \ac{HE} to every single micro-interaction is computationally infeasible for real-time monitoring. This necessitates a hybrid approach where \ac{HE} is reserved for critical decision points.}
\end{itemize}

There is currently no unified framework that effectively balances these conflicting requirements. This research seeks to bridge this gap by proposing an \textbf{adaptive dual-mode architecture} that utilizes \textbf{Orthogonal Random Projections} for lightweight, continuous monitoring and reserves \ac{HE} for high-security checkpoints, ensuring privacy without degrading the speed required for edge devices.

\subsection{Research Aim and Objectives}

\paragraph{} The primary aim of this research is to develop a lightweight, privacy-preserving framework for continuous behavioral authentication on \textbf{resource-constrained edge devices}. This framework seeks to balance recognition accuracy and system latency by utilizing Orthogonal Random Projections for template security and \ac{Deep SVDD} for efficient anomaly detection.

\paragraph{}The technical feasibility of utilizing Orthogonal Random Projections and Deep SVDD for anomaly detection has been validated through a functional code prototype. Preliminary execution results demonstrating the successful separation of genuine user patterns from impostors are detailed in Appendix A.

Preliminary results for this objective are documented in Appendix \ref{App:A}.


\begin{enumerate}
    \item \textbf{To design a privacy-preserving feature transformation pipeline:} Develop a mechanism using \textbf{Orthogonal Random Projections (\ac{JL} Lemma)} \cite{johnson1984extensions} that secures behavioral biometric templates (making them mathematically irreversible) while preserving the Euclidean distances required for accurate pattern recognition.
    
    \item \textbf{To optimize feature engineering for mobile efficiency:} Implement \textbf{KDPrint-style standardization} to transform raw time-series data into standardized image encodings, ensuring high recognition accuracy without the noise sensitivity of Min-Max scaling \cite{kim2024kdprint}.
    
    \item \textbf{To implement a lightweight anomaly detection model:} Develop a \textbf{\ac{Deep SVDD}} classifier \cite{ruff2018deepsvdd} capable of running offline on \textbf{mid-range edge devices} to distinguish between genuine users and imposters with minimal computational overhead.
    
    \item \textbf{To evaluate the trade-off between privacy, accuracy, and latency:} Conduct a comparative analysis of the proposed framework against existing baselines (such as raw-data \acp{RNN} and \ac{HE}), measuring performance metrics including \textbf{\ac{EER}}, \textbf{System Latency (ms)}, and \textbf{Memory Usage} \cite{kiyani2020continuous, ruff2018deepsvdd}.
\end{enumerate}

\subsection{Research Questions}

To address the identified gaps in privacy and efficiency, this research aims to answer the following key questions:

\begin{enumerate}
    \item \textbf{Primary Research Question:} To what extent can \textbf{Orthogonal Random Projections} (based on the \ac{JL} Lemma) balance the conflicting requirements of template privacy, recognition accuracy, and system latency in behavioral authentication? 
    \item \textbf{Impact on Privacy and Irreversibility:} How effective is the proposed projection mechanism in rendering behavioral templates mathematically irreversible to attackers, compared to storing raw features or using standard Min-Max scaling? 
    
    \item \textbf{Feasibility for Resource-Constrained Environments:} Can a \textbf{\ac{Deep SVDD}} anomaly detection model achieve real-time authentication latency (e.g., $<200$ms) on \textbf{resource-constrained edge devices} without exceeding memory constraints? 
\end{enumerate}

\newpage % 1.0 Introduction
% --- sections/2_background.tex ---

% --- Section 2.0: Background ---
% [BACKGROUND] ensures it appears as uppercase in the Table of Contents
\section[BACKGROUND]{Background}

% This section provides the context for your research. 
% It should introduce the broader field and why this area is significant.

\subsection{Technical and Theoretical Background}
\subsubsection{Keystroke Dynamics}
\paragraph{} This is the measurement of biomechanical typing patterns. The fundamental features include Dwell Time (duration a key is pressed) and Flight Time (latency between releasing one key and pressing the next). These features form a unique "digital signature" for each user.


\subsubsection{Mouse Dynamics}
\paragraph{} This involves analyzing the unique behavioral patterns of a user's mouse interactions. Unlike simple click-tracking, this research focuses on complex motor-skill features:
\begin{itemize}
    \item \textbf{Movement Efficiency:} The ratio of the straight-line distance to the actual path taken (analyzing hand jitter and curvature).
    \item \textbf{Velocity \& Acceleration Profiles:} The rate of speed change as the cursor approaches a target (e.g., users often decelerate differently when clicking a button).
    \item \textbf{Click-to-Click Latency:} The timing between releasing a button and moving to the next location.
    \item \textbf{Drag-and-Drop characteristics::} The pressure and speed consistency during sustained click events.
\end{itemize}


\subsubsection{Recurrent Neural Networks (RNNs) \& LSTMs}
\paragraph{} Since keystroke data is inherently sequential (a time-series of events), standard feed-forward networks often fail to capture temporal dependencies. RNNs, and specifically Long Short-Term Memory (LSTM) networks, are theoretically suited for this task as they maintain a "memory" of previous inputs, allowing them to model complex, non-linear typing rhythms over time.

\subsubsection{Johnson-Lindenstrauss (JL) Lemma} 
\paragraph{} To address the "curse of dimensionality" and system latency, this research utilizes the JL Lemma. This mathematical theorem states that points in a high-dimensional space can be projected into a lower-dimensional space using Orthogonal Random Projections while approximately preserving the Euclidean distances between them. This allows for lightweight processing without significant loss of accuracy.

\subsubsection{Homomorphic Encryption (HE)} 
\paragraph{}  To ensure privacy, the system employs Homomorphic Encryption, a cryptographic form that allows computations to be performed on encrypted data without first decrypting it. This ensures that the user's raw biometric template is never exposed in plaintext during the authentication process.

% Discuss the fundamental theories and technologies relevant to your work.
% For example, explain the basics of RNNs, Behavioral Biometrics, or Authentication protocols.
% Use citations to support your definitions.

\subsection{Literature Review}

\subsubsection{Overview of Reviewed Literature}

To establish a theoretical framework for this research, a comprehensive review of existing literature was conducted, focusing on Keystroke Dynamics, Mouse Dynamics, and Privacy-Preserving Machine Learning. The following table summarizes the key research papers referenced, highlighting their specific contributions to this project (``Key Takeaway'') and the limitations (``Research Gap'') that this proposal aims to address.

% [H] forces the table to stay exactly here
\begin{table}[H] 
    \centering
    \caption{Summary of Key Related Studies}
    \label{tab:lit_review_summary}
    \renewcommand{\arraystretch}{1.5} 
    \begin{tabular}{|p{0.2\textwidth}|p{0.37\textwidth}|p{0.37\textwidth}|}
        \hline
        \textbf{Reference \& Study} & \textbf{Key Contribution to This Project} & \textbf{Identified Gap / Limitation} \\
        \hline
        Gaines et al. (1980) \& Joyce et al. (1990) & 
        \textbf{Foundational Theory:} Established that typing rhythms (dwell/flight time) are unique and stable enough for identity verification. & 
        Relied on static, fixed-text passwords, which are insufficient for continuous authentication. \\
        \hline
        Mondal \& Bours (2017) & 
        \textbf{Multimodal Fusion:} Proved that combining Keystroke and Mouse dynamics significantly reduces Equal Error Rate (EER) compared to single modalities. & 
        Fusion was achieved by simply concatenating features, creating a high-dimensional vector that slows down real-time processing. \\
        \hline
        Kim et al. (2018) & 
        \textbf{Feature Engineering:} Introduced ``user-adaptive'' features, showing that personalized feature selection improves accuracy. & 
        Focused entirely on accuracy; lacked any ``template protection'' or encryption to secure the stored data. \\
        \hline
        \textbf{Kim et al. (2025)} & 
        \textbf{Deep SVDD Validation:} Demonstrated that Deep SVDD outperforms traditional models (6.7\% EER) by encoding time-series data into images. & 
        Restricted to \textit{mobile PINs} (touch interactions) and lacked cryptographic encryption (HE) or multimodal fusion (Mouse). \\
        \hline
        Kiyani et al. (2020) & 
        \textbf{Continuous Authentication:} Validated the use of Recurrent Neural Networks (RNNs) for verifying users continuously, not just at login. & 
        Did not address the high latency introduced when trying to add encryption to these continuous streams. \\
        \hline
        Islam et al. (2021) & 
        \textbf{Scalability Metrics:} Provided a framework for measuring how error rates grow as the user database size increases. & 
        Addressed scalability of accuracy but not the scalability of privacy (how to store millions of secure templates). \\
        \hline
    \end{tabular}
\end{table}

\subsubsection{Foundational Studies (Static Authentication)}
\paragraph{} Early research laid the groundwork for typing pattern analysis. Gaines et al. (1980) demonstrated that typing rhythms are unique to individuals, identifying that even simple digraph latencies could distinguish users with high confidence. Building on this, Joyce and Gupta (1990) formalized the use of "latency signatures" for identity verification, showing that comparing a test signature against a mean reference signature could achieve low impostor pass rates. Monrose et al. (1999) extended this concept to "password hardening," combining typing rhythms with passwords to increase entropy against offline attacks.

\subsubsection{Feature Engineering and Adaptive Systems}
\paragraph{} As research matured, the focus shifted to handling the variability in human behavior. Kim et al. (2018) highlighted the limitations of fixed-text authentication and proposed "user-adaptive feature extraction." Their work demonstrated that by dynamically selecting features based on a user's specific typing speed ranks (rather than a fixed keyboard layout), the Equal Error Rate (EER) could be significantly reduced. This underscored the need for personalized models in behavioral authentication.

\subsubsection{Mouse Dynamics and Multimodal Fusion}
\paragraph{} Research into mouse dynamics has evolved from simple statistics to complex trajectory analysis. Ahmed and Traore (2007) pioneered the use of movement speed and direction histograms, achieving reasonable accuracy but noting high false-positive rates in unconstrained environments. Later, Zheng et al. (2011) improved this by analyzing "point-by-point" angle-based metrics, proving that fine-grained motor skills are harder to forge than simple click statistics.
\paragraph{} Recent studies have shifted toward Multimodal Authentication, combining keystroke and mouse data to overcome the weaknesses of single-modality systems. Mondal and Bours (2017) demonstrated that fusing these two biometrics significantly reduces the Equal Error Rate (EER) because an imposter is unlikely to mimic both a user's typing rhythm and their mouse hand-eye coordination simultaneously. However, existing multimodal frameworks often simply concatenate feature vectors, leading to massive dimensionality that slows down real-time processing—a problem this research aims to solve using the JL Lemma.

\subsubsection{Deep Learning and Continuous Authentication}
\paragraph{} Recent approaches have adopted Deep Learning (DL) to improve accuracy. Zareen et al. (2018) utilized Bayesian Regularized Neural Networks, achieving an EER of 0.9\%, which outperformed many standard statistical methods. Kiyani et al. (2020) proposed a "Robust Recurrent Confidence Model" (R-RCM) using ensemble learning for continuous monitoring. 

\paragraph{} Most recently, \textbf{Kim et al. (2025)} proposed "KDPrint," a method that transforms keystroke dynamics into images and utilizes \textbf{Deep Support Vector Data Description (Deep SVDD)} for anomaly detection. Their work achieved a notable EER of 6.7\% on mobile devices, validating Deep SVDD as a superior classifier for one-class authentication tasks. However, their approach focused exclusively on mobile PINs (touch interaction) and utilized image encoding for feature representation rather than cryptographic privacy. This leaves a significant gap for applying Deep SVDD to \textit{encrypted, multimodal} desktop environments, which this research aims to fulfill.

\subsubsection{Scalability and System Performance} 
\paragraph{} While accuracy has improved, scalability remains a challenge. Islam et al. (2021) introduced the notion of "Scalable Behavioral Authentication," analyzing how verification error rates increase as the user database grows. They proposed a "doppelganger-based" personalized verification algorithm to mitigate error growth, highlighting that system performance must be evaluated not just on accuracy, but on its ability to handle large-scale deployments.


% critically analyze existing work. 
% Do not just list papers; group them by themes (e.g., "Previous Approaches to Latency", "Privacy Methods").
% Highlight what they achieved and where they fell short.

\subsection{Research Gap}

\paragraph{} Despite the extensive literature on improving the accuracy of behavioral biometrics (Kiyani et al., 2020; Zareen et al., 2018) and recent advancements in Deep Learning anomaly detection (Kim et al., 2025), a critical trilemma remains unsolved: balancing \textbf{Accuracy}, \textbf{Efficiency}, and \textbf{Privacy}.

\begin{enumerate}
    \item \textbf{Computation vs. Latency in Multimodal Systems:} Integrating Mouse Dynamics with Keystroke Dynamics doubles the feature space complexity. While recent studies like Kim et al. (2025) successfully utilized Deep SVDD for mobile PINs, they focused on single-modality touch data. Processing a high-dimensional, multimodal stream (typing + mouse trajectories) creates a computational bottleneck that current frameworks fail to address efficiently without dimensionality reduction.
    
    \item \textbf{Privacy Vulnerability (Lack of Encryption):} Most existing frameworks focus on verifying raw feature vectors or transformed representations. For instance, while Kim et al. (2025) introduced ``image encoding'' to obscure raw keystrokes, this is a feature transformation technique, not a cryptographic privacy guarantee. It lacks the mathematical irreversibility of \textbf{Homomorphic Encryption (HE)}. If the central database is compromised, these behavioral templates (images or vectors) are susceptible to reverse-engineering or replay attacks, leading to a permanent loss of digital identity.
    
    \item \textbf{Lack of Integrated Privacy-Preserving Architectures:} While cryptographic solutions exist, standard encryption prevents the system from performing the distance calculations needed for authentication (like Euclidean distance). There is currently no implementation that combines the anomaly detection power of \textbf{Deep SVDD} (validated by Kim et al., 2025) with \textbf{Homomorphic Encryption} for secure, privacy-preserving inference.
\end{enumerate}

\paragraph{Gap Definition:} There is currently no unified framework that utilizes \textbf{Orthogonal Random Projections (JL Lemma)} to compress the combined feature space of both Keystroke and Mouse Dynamics for lightweight processing, while simultaneously preserving privacy using \textbf{Homomorphic Encryption}. Unlike Kim et al. (2025), which applies Deep SVDD to unencrypted mobile data, this study aims to bridge the gap by creating a fast, \textit{cryptographically secure}, multimodal authentication system for desktop environments.

% Based on your literature review, clearly state what is missing.
% This justifies why your specific research is necessary.
% Example: "While previous studies focused on accuracy, few addressed the latency constraints on mobile devices..."

\subsection{Assumptions and Constraints}

% List the limitations and scope of your project.
\subsubsection{Assumptions}
    
    \begin{itemize}
        \item It is assumed that the user’s typing behavior is relatively stable over short periods but may exhibit gradual "concept drift" which the model must accommodate.
        \item It is assumed that the users are utilizing standard physical keyboards; virtual/touchscreen keyboards are outside the scope of this specific study (unless specified otherwise).
        \item It is assumed that the mouse data collection frequency (e.g., 50Hz) is sufficient to capture micro-movements without overwhelming the system bus.
        \item The "trust" of the endpoint device (personal laptop) is assumed for the initial data capture phase before encryption.
    \end{itemize}
\subsubsection{Constraints}
   
    \begin{itemize}
        \item \textbf{Hardware Limitations:} The proposed model must be lightweight enough to run on standard consumer hardware (e.g., a laptop with a mid-range GPU like an RTX 3050) without causing noticeable input lag.
        \item \textbf{Data Availability:} The research is constrained by the need to collect a custom dataset or use public datasets that may not perfectly reflect the specific "free-text" behavior required for continuous authentication.
        \item \textbf{Encryption Overhead:} Homomorphic Encryption introduces significant computational overhead. The system is constrained to optimize this trade-off to ensure the authentication decision happens within a usable timeframe (e.g., under 200ms per window).
    \end{itemize}
   


\newpage   % 2.0 Background
% --- sections/3_methodology.tex ---

% --- Section 3.0: Methodology ---
% [METHODOLOGY AND PROJECT DESIGN] ensures it appears uppercase in the Table of Contents
\section[METHODOLOGY AND PROJECT DESIGN]{Methodology and Project Design}

% This section outlines how you will conduct your research.
% It should be detailed enough that another researcher could replicate your study.

\subsection{Overview of the Proposed Methodology/Research Design}

% Provide a high-level summary of your approach.
% You might want to include a system architecture diagram here later.
% Explain whether this is a quantitative, qualitative, or mixed-method study.

\subsection{Data Collection}

% Describe your data sources.
% Are you using a public dataset (e.g., HMOG, UCI)? 
% Or are you collecting your own data? If so, describe the sensors and sampling rate.

\subsection{Ethical Considerations}

% Discuss any ethical issues related to your data or subjects.
% For behavioral biometrics, privacy is a key concern.
% Mention if you have or need IRB approval.

\subsection{Evaluation and Validation}

% How will you measure success?
% metric examples:
\begin{itemize}
    \item \textbf{False Acceptance Rate (FAR)}
    \item \textbf{False Rejection Rate (FRR)}
    \item \textbf{Equal Error Rate (EER)}
    \item \textbf{System Latency (ms)}
\end{itemize}  % 3.0 Methodology
% --- sections/4_results.tex ---

% --- Section 4.0: Anticipated Results ---
% [ANTICIPATED RESULTS/FINAL PRODUCTS] ensures it appears uppercase in the Table of Contents
\section[ANTICIPATED RESULTS/FINAL PRODUCTS]{Anticipated Results/Final Products}

% This section describes what you expect to achieve by the end of the project.
% It connects back to your objectives.

\subsection{Expected Outcomes}

% Describe the tangible and intangible results.
% Example:
\begin{itemize}
    \item A fully functional Android application capable of continuous authentication.
    \item A comparative analysis report on different RNN architectures (LSTM vs. GRU).
    \item A research paper submitted to an IEEE conference.
\end{itemize}

\subsection{Project Deliverables}

% List the concrete items you will submit.
\begin{enumerate}
    \item **Software Prototype:** Source code and executable APK.
    \item **Final Thesis:** Comprehensive documentation of the research process.
    \item **User Manual:** Guide for installing and testing the application.
\end{enumerate}

\subsection{Project Timeline}

% Briefly mention your schedule or refer to a Gantt chart if you have one.
% "The project will follow a 6-month timeline, starting with literature review..."      % 4.0 Anticipated Results/Final Products
% --- sections/5_project_schedule.tex ---

% --- Section 5.0: Project Schedule ---
% [PROJECT SCHEDULE] ensures it appears uppercase in the Table of Contents
\section[PROJECT SCHEDULE]{Project Schedule}

% This section outlines the timeline for your research.
% You can use a table or insert an image of a Gantt chart here.

\subsection{Timeline and Gantt Chart}

% Describe the phases of your project (e.g., Literature Review, Implementation, Testing).
% Example table for a simple schedule:
\begin{table}[h!]
    \centering
    \begin{tabular}{|l|l|l|}
        \hline
        \textbf{Phase} & \textbf{Activity} & \textbf{Duration} \\ \hline
        1 & Literature Review & Months 1-2 \\ \hline
        2 & Data Collection & Month 3 \\ \hline
        3 & Model Implementation & Months 4-5 \\ \hline
        4 & Testing and Validation & Month 6 \\ \hline
        5 & Thesis Writing & Months 7-8 \\ \hline
    \end{tabular}
    \caption{Proposed Project Timeline}
    \label{tab:schedule}
\end{table}

\subsection{Milestones and Deliverables}

% List key dates where specific parts of the project must be finished.
\begin{itemize}
    \item \textbf{Milestone 1:} Completion of Literature Review (Date)
    \item \textbf{Milestone 2:} Prototype Development (Date)
    \item \textbf{Final Submission:} Thesis and Defense (Date)
\end{itemize} % 5.0 Project Schedule

% --- 5. BIBLIOGRAPHY (IEEE Style) ---
\clearpage

% --- REFERENCES ---
% 1. Define a custom style: Title Case for Body, UPPERCASE for TOC
\defbibheading{upperbib}{%
    \section*{References} % The '*' makes it unnumbered
    \addcontentsline{toc}{section}{REFERENCES} 
}
% 2. Print the bibliography using that style
\printbibliography[heading=upperbib]

% --- APPENDICES ---
\appendix
\section{Appendices}

\subsection{Appendix A: JL Projection + Deep SVDD Prototype} \label{App:A}
This appendix contains the Python implementation of the core framework using synthetic data to validate the anomaly detection logic and dimensionality reduction.

\begin{lstlisting}[language=Python, caption=JL Projection and Deep SVDD Implementation]
import numpy as np
import torch
import torch.nn as nn
import torch.optim as optim
from sklearn.random_projection import GaussianRandomProjection

# ==========================================
# 1. Settings and Data Generation
# ==========================================
np.random.seed(42)
torch.manual_seed(42)

original_dim = 100     # Original feature dimension
projected_dim = 50     # Dimension after JL projection

# --- User Data (Target User) ---
base_pattern = np.random.rand(1, original_dim)
user_patterns = base_pattern + np.random.normal(0, 0.05, (10, original_dim))

# --- Imposter Data ---
imposter_patterns = np.random.rand(10, original_dim)

# Combine all data
all_raw_data = np.vstack((user_patterns, imposter_patterns))
print(f"Original Data Shape: {all_raw_data.shape}")

# ==========================================
# 2. Privacy Preservation using JL Projection
# ==========================================
transformer = GaussianRandomProjection(
    n_components=projected_dim,
    random_state=42
)

all_projected_data = transformer.fit_transform(all_raw_data)

print(f"Projected Data Shape: {all_projected_data.shape}")
print("-" * 50)

# ==========================================
# 3. Train/Test Split
# ==========================================
# User data: first 5 for training, next 5 for testing
X_train_np = all_projected_data[:5]
X_test_user_np = all_projected_data[5:10]

# Imposter data: used only for testing
X_test_imposter_np = all_projected_data[10:]

# Convert to PyTorch tensors
X_train = torch.tensor(X_train_np, dtype=torch.float32)
X_test_user = torch.tensor(X_test_user_np, dtype=torch.float32)
X_test_imposter = torch.tensor(X_test_imposter_np, dtype=torch.float32)

# Full test set
X_test_all = torch.cat((X_test_user, X_test_imposter), dim=0)

# ==========================================
# 4. Deep SVDD Model Definition
# ==========================================
class DeepSVDD(nn.Module):
    def __init__(self, input_dim):
        super(DeepSVDD, self).__init__()
        self.encoder = nn.Sequential(
            nn.Linear(input_dim, 32),
            nn.ReLU(),
            nn.Linear(32, 16)  # Latent representation
        )

    def forward(self, x):
        return self.encoder(x)

# Initialize model and optimizer
model = DeepSVDD(input_dim=projected_dim)
optimizer = optim.Adam(model.parameters(), lr=0.001)

# Initialize center (c) as mean of training embeddings
with torch.no_grad():
    c = torch.mean(model(X_train), dim=0)

# ==========================================
# 5. Training Phase (One-Class Learning)
# ==========================================
print("Training Model...")
epochs = 300
model.train()

for epoch in range(epochs):
    optimizer.zero_grad()
    outputs = model(X_train)

    # Loss = mean squared distance to center
    dist = torch.sum((outputs - c) ** 2, dim=1)
    loss = torch.mean(dist)

    loss.backward()
    optimizer.step()

print("Training Complete.")

# ==========================================
# 6. Radius (R) Determination
# ==========================================
model.eval()
with torch.no_grad():
    train_outputs = model(X_train)
    train_dists = torch.sum((train_outputs - c) ** 2, dim=1)

    max_train_dist = torch.max(train_dists).item()

    # Add safety margin
    margin = 0.05
    radius = max_train_dist + margin

print("\n[Configuration]")
print(f"  Max Train Dist : {max_train_dist:.4f}")
print(f"  Safety Margin  : {margin:.4f}")
print(f"  Final Radius(R): {radius:.4f}")
print("-" * 50)

# ==========================================
# 7. Testing and Evaluation
# ==========================================
print(f"{'Sample Type':<20} | {'Distance':<10} | {'Status':<12} | {'Result'}")
print("-" * 65)

with torch.no_grad():

    # User Test Samples
    user_outputs = model(X_test_user)
    user_dists = torch.sum((user_outputs - c) ** 2, dim=1)

    for i, dist in enumerate(user_dists):
        d_val = dist.item()
        status = "Authorized" if d_val <= radius else "Blocked"
        result = "PASS" if d_val <= radius else "False Reject"
        print(f"User (Genuine) {i+1:<5} | {d_val:.4f}     | {status:<12} | {result}")

    print("-" * 65)

    # Imposter Test Samples
    imposter_outputs = model(X_test_imposter)
    imposter_dists = torch.sum((imposter_outputs - c) ** 2, dim=1)

    for i, dist in enumerate(imposter_dists):
        d_val = dist.item()
        status = "Authorized" if d_val <= radius else "Blocked"
        result = "PASS" if d_val > radius else "False Accept"
        print(f"Imposter {i+1:<11} | {d_val:.4f}     | {status:<12} | {result}")

\end{lstlisting}

\subsubsection{Prototype Execution Output and Observations}
The following table represents the console output from the execution of the Deep SVDD prototype. The results validate that the system successfully projected a 100-dimensional feature space into 50 dimensions while maintaining the distance integrity required to distinguish between genuine users and impostors

\begin{lstlisting}[language=bash, frame=single]
Original Data Shape: (20, 100)
Projected Data Shape: (20, 50)
--------------------------------------------------
Training Model...
Training Complete.

[Configuration]
  Max Train Dist : 0.0000
  Safety Margin  : 0.0500
  Final Radius(R): 0.0500
--------------------------------------------------
Sample Type          | Distance   | Status       | Result
-----------------------------------------------------------------
User (Genuine) 1     | 0.0112     | Authorized   | PASS
User (Genuine) 2     | 0.0023     | Authorized   | PASS
User (Genuine) 3     | 0.0050     | Authorized   | PASS
User (Genuine) 4     | 0.0022     | Authorized   | PASS
User (Genuine) 5     | 0.0044     | Authorized   | PASS
-----------------------------------------------------------------
Imposter 1           | 0.2351     | Blocked      | PASS
Imposter 2           | 0.1823     | Blocked      | PASS
Imposter 3           | 0.2035     | Blocked      | PASS
Imposter 4           | 0.2297     | Blocked      | PASS
Imposter 5           | 0.0924     | Blocked      | PASS
... (Truncated for brevity) ...
\end{lstlisting}

\textbf{Analysis:} The prototype achieved a 100\% success rate on synthetic samples. The user distances remained well within the learned radius ($R=0.0500$), while all impostor samples were significantly outside the hypersphere, validating the "one-class" classification approach.



\subsection{Appendix B: SVM Baseline Classifier} \label{App:B}
This appendix provides a baseline comparison using a traditional Support Vector Machine (SVM) on a small-scale behavioral dataset.

\begin{lstlisting}[language=Python, caption=SVM Baseline Classifier]
import numpy as np
from sklearn.svm import SVC
from sklearn.metrics import accuracy_score, classification_report
from sklearn.preprocessing import StandardScaler

# ==========================================
# 1. Data Entry
# Features: [dwell_avg, flight_avg, traj_avg]
# ==========================================

# User Data (Label = 1)
user_data = np.array([
    [0.093341, 0.364395, 681.6144],
    [0.085055, 0.355090, 596.1330],
    [0.091337, 0.428217, 663.5182],
    [0.091395, 0.306243, 580.3732],
    [0.087598, 0.401027, 614.3611],
    [0.091835, 0.358024, 681.0282],
    [0.087437, 0.317831, 664.8624],
    [0.097054, 0.330271, 493.6987],
    [0.091275, 0.401341, 579.4574],
    [0.095933, 0.361527, 500.6159]
])

# Imposter Data (Label = 0)
imposter_data = np.array([
    [0.090275, 0.521462, 516.0034],
    [0.100985, 0.833044, 412.5477],
    [0.073261, 0.687610, 663.6120],
    [0.130867, 0.897945, 290.1982],
    [0.179208, 0.670023, 345.4835],
    [0.100080, 0.849405, 273.3659],
    [0.126100, 0.247867, 401.4568],
    [0.076832, 0.466030, 310.1983],
    [0.067409, 0.853341, 409.9432],
    [0.089729, 0.431692, 669.1964]
])

# ==========================================
# 2. Train/Test Split (6 Training / 4 Testing)
# ==========================================

# Training Data
X_train = np.vstack((user_data[:6], imposter_data[:6]))
y_train = np.array([1] * 6 + [0] * 6)  # 1 = User, 0 = Imposter

# Testing Data
X_test = np.vstack((user_data[6:], imposter_data[6:]))
y_test = np.array([1] * 4 + [0] * 4)

print(f"Training Data: {len(X_train)} samples")
print(f"Testing Data:  {len(X_test)} samples")
print("-" * 40)

# ==========================================
# 3. Feature Scaling
# ==========================================
# Standardization ensures all features are on a similar scale

scaler = StandardScaler()
X_train_scaled = scaler.fit_transform(X_train)
X_test_scaled = scaler.transform(X_test)

# ==========================================
# 4. Model Training (Linear SVM)
# ==========================================

model = SVC(kernel='linear')
model.fit(X_train_scaled, y_train)

# ==========================================
# 5. Testing and Evaluation
# ==========================================

predictions = model.predict(X_test_scaled)

print("Actual Labels:   ", y_test)
print("Predicted Labels:", predictions)
print("-" * 40)

correct = 0
for i in range(len(y_test)):
    actual = "User" if y_test[i] == 1 else "Imposter"
    predicted = "User" if predictions[i] == 1 else "Imposter"

    status = "PASS" if y_test[i] == predictions[i] else "FAIL"
    if y_test[i] == predictions[i]:
        correct += 1

    print(f"Sample {i+1} (Actual: {actual}) --> Predicted: {predicted} | {status}")

print("-" * 40)
print(f"Accuracy: {correct}/{len(y_test)} ({(correct/len(y_test))*100}%)")

\end{lstlisting}

\subsubsection{Baseline Performance Output}
The execution results for the SVM baseline highlight the limitations of traditional binary classifiers when dealing with limited behavioral samples.

\begin{lstlisting}[language=bash, frame=single]
Training Data: 12 samples
Testing Data:  8 samples
----------------------------------------
Actual Labels:    [1 1 1 1 0 0 0 0]
Predicted Labels: [1 1 1 1 1 0 0 1]
----------------------------------------
Sample 1 (Actual: User) --> Predicted: User | PASS
...
Sample 5 (Actual: Imposter) --> Predicted: User | FAIL
Sample 8 (Actual: Imposter) --> Predicted: User | FAIL
----------------------------------------
Accuracy: 6/8 (75.0%)
\end{lstlisting}

\textbf{Analysis:} The 75\% accuracy rate and the "False Accept" errors in samples 5 and 8 illustrate why a more robust, deep-learning-based anomaly detector (Deep SVDD) is proposed for the final system to improve security against sophisticated impostors.

\end{document}