% --- sections/2_background.tex ---

% --- Section 2.0: Background ---
% [BACKGROUND] ensures it appears as uppercase in the Table of Contents
\section[BACKGROUND]{Background}

% This section provides the context for your research. 
% It should introduce the broader field and why this area is significant.

\subsection{Technical and Theoretical Background}

% Discuss the fundamental theories and technologies relevant to your work.
% For example, explain the basics of RNNs, Behavioral Biometrics, or Authentication protocols.
% Use citations to support your definitions.

\subsection{Literature Review}

% critically analyze existing work. 
% Do not just list papers; group them by themes (e.g., "Previous Approaches to Latency", "Privacy Methods").
% Highlight what they achieved and where they fell short.

\subsection{Research Gap}

% Based on your literature review, clearly state what is missing.
% This justifies why your specific research is necessary.
% Example: "While previous studies focused on accuracy, few addressed the latency constraints on mobile devices..."

\subsection{Assumptions and Constraints}

% List the limitations and scope of your project.
\begin{itemize}
    \item \textbf{Assumptions:} The system assumes users have a smartphone with standard accelerometer and gyroscope sensors.
    \item \textbf{Constraints:} The model must run offline on mid-range mobile devices without exceeding 50MB of memory usage.
\end{itemize}