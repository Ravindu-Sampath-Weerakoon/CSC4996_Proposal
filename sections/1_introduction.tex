\section[INTRODUCTION]{Introduction}

\paragraph{Overview of the Research Domain}
In mechanisms such as \ac{PIN} and passwords are increasingly becoming single points of failure \cite{pirzado2021keystroke}. They are susceptible to vulnerabilities like shoulder surfing, brute-force attacks, and social engineering \cite{pirzado2021keystroke, kiyani2020continuous}. To address these vulnerabilities, Behavioral Biometrics---specifically Keystroke Dynamics and Mouse/Touch Dynamics---has emerged as a powerful alternative \cite{rahman2021scalable, kim2024kdprint}. Unlike static passwords, behavioral biometrics allow for continuous, passive authentication, verifying a user's identity based on \textit{how} they interact with a device rather than \textit{what} they know \cite{shepherd1995continuous, monrose1999hardening}.

\subsection{Problem Statement}
The primary problem addressed by this research is the critical trade-off between \textbf{user privacy} and \textbf{system latency} in behavioral authentication systems on \textbf{resource-constrained edge devices}. While behavioral biometrics---such as keystroke dynamics and touch interactions---offer a robust solution for continuous authentication, the storage and processing of these behavioral patterns present significant security risks \cite{rahman2021scalable}.

This issue affects millions of users across \textbf{heterogeneous platforms, ranging from high-end desktops to resource-constrained mobile and IoT devices}. Unlike passwords, behavioral biometrics are immutable; a user cannot change their typing rhythm or hand geometry if the biometric template is compromised. Therefore, a breach of raw behavioral data constitutes a permanent loss of digital identity \cite{pirzado2021keystroke}.

Current approaches fail to address this problem effectively due to a technical dichotomy between accuracy and efficiency:

\begin{itemize}
    \item \textbf{Privacy Gaps in High-Accuracy Models:} State-of-the-art \ac{DL} frameworks, such as those utilizing \acp{RNN} or Image-based Encoding (e.g., KDPrint), achieve low \ac{EER} but typically require the storage of \textbf{raw behavioral features} \cite{kim2024kdprint, kiyani2020continuous}. This creates a single point of failure where the database becomes a high-value target for attackers.
    
    \item \textbf{Efficiency Gaps in Privacy-Preserving Methods:} Conversely, strong cryptographic solutions like \ac{FHE} allow for secure computation but suffer from prohibitive computational overhead. Research indicates that such heavy \ac{HE} schemes often introduce latencies ranging from seconds to minutes, rendering them impractical for \textbf{real-time, continuous authentication} where decisions must be made in milliseconds to maintain a seamless user experience \cite{rahman2021scalable}. \textbf{Therefore, applying \ac{HE} to every single micro-interaction is computationally infeasible for real-time monitoring. This necessitates a hybrid approach where \ac{HE} is reserved for critical decision points.}
\end{itemize}

There is currently no unified framework that effectively balances these conflicting requirements. This research seeks to bridge this gap by proposing an \textbf{adaptive dual-mode architecture} that utilizes \textbf{Orthogonal Random Projections} for lightweight, continuous monitoring and reserves \ac{HE} for high-security checkpoints, ensuring privacy without degrading the speed required for edge devices.

\subsection{Research Aim and Objectives}

\paragraph{} The primary aim of this research is to develop a lightweight, privacy-preserving framework for continuous behavioral authentication on \textbf{resource-constrained edge devices}. This framework seeks to balance recognition accuracy and system latency by utilizing Orthogonal Random Projections for template security and \ac{Deep SVDD} for efficient anomaly detection.

\paragraph{}The technical feasibility of utilizing Orthogonal Random Projections and Deep SVDD for anomaly detection has been validated through a functional code prototype. Preliminary execution results demonstrating the successful separation of genuine user patterns from impostors are detailed in Appendix A.

Preliminary results for this objective are documented in Appendix \ref{App:A}.


\begin{enumerate}
    \item \textbf{To design a privacy-preserving feature transformation pipeline:} Develop a mechanism using \textbf{Orthogonal Random Projections (\ac{JL} Lemma)} \cite{johnson1984extensions} that secures behavioral biometric templates (making them mathematically irreversible) while preserving the Euclidean distances required for accurate pattern recognition.
    
    \item \textbf{To optimize feature engineering for mobile efficiency:} Implement \textbf{KDPrint-style standardization} to transform raw time-series data into standardized image encodings, ensuring high recognition accuracy without the noise sensitivity of Min-Max scaling \cite{kim2024kdprint}.
    
    \item \textbf{To implement a lightweight anomaly detection model:} Develop a \textbf{\ac{Deep SVDD}} classifier \cite{ruff2018deepsvdd} capable of running offline on \textbf{mid-range edge devices} to distinguish between genuine users and imposters with minimal computational overhead.
    
    \item \textbf{To evaluate the trade-off between privacy, accuracy, and latency:} Conduct a comparative analysis of the proposed framework against existing baselines (such as raw-data \acp{RNN} and \ac{HE}), measuring performance metrics including \textbf{\ac{EER}}, \textbf{System Latency (ms)}, and \textbf{Memory Usage} \cite{kiyani2020continuous, ruff2018deepsvdd}.
\end{enumerate}

\subsection{Research Questions}

To address the identified gaps in privacy and efficiency, this research aims to answer the following key questions:

\begin{enumerate}
    \item \textbf{Primary Research Question:} To what extent can \textbf{Orthogonal Random Projections} (based on the \ac{JL} Lemma) balance the conflicting requirements of template privacy, recognition accuracy, and system latency in behavioral authentication? 
    \item \textbf{Impact on Privacy and Irreversibility:} How effective is the proposed projection mechanism in rendering behavioral templates mathematically irreversible to attackers, compared to storing raw features or using standard Min-Max scaling? 
    
    \item \textbf{Feasibility for Resource-Constrained Environments:} Can a \textbf{\ac{Deep SVDD}} anomaly detection model achieve real-time authentication latency (e.g., $<200$ms) on \textbf{resource-constrained edge devices} without exceeding memory constraints? 
\end{enumerate}

\newpage